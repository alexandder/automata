\documentclass{xmgr}
% Jeśli nowe rozdziały mają się zaczynać na stronach
% nieparzystych:
%\documentclass[openright]{xmgr}

%\defaultfontfeatures{Scale=MatchLowercase}
%\setmainfont[Numbers=OldStyle,Ligatures=TeX]{Minion Pro}
%\setsansfont[Numbers=OldStyle,Ligatures=TeX]{Myriad Pro}
% for fontspec version < 2.0
%\setmainfont[Numbers=OldStyle,Mapping=tex-text]{Times}
%\setsansfont[Numbers=OldStyle,Mapping=tex-text]{Times}
%\setmonofont[Scale=0.75]{Monaco}

% Opcjonalnie identyfikator dokumentu
% drukowany tylko z włączoną opcją 'brudnopis':
\wersja   {wersja wstępna [\ymdtoday]}

\author   {Aleksander Bołt}
\nralbumu {186565}


\title    {Dynamiczne własności jednowymiarowych automatów komórkowych}
\date     {2016}
\miejsce  {Gdańsk}

\opiekun  { dr Barbara Wolnik}

% dodatkowe polecenia
%\renewcommand{\filename}[1]{\texttt{#1}}
%\definecolor{stress}{cmyk}{0,1,0.13,0} % RubineRed
%\definecolor{topic}{cmyk}{0.98,0.13,0,0.43} % MidnightBlue
\newtheorem{theorem}{Twierdzenie}
\newtheorem{corollary}{Wniosek}
\newtheorem{definition}{Definicja}
\newtheorem{properties}{Własności}
\newtheorem{example}{Przykład}
\newtheorem{lemma}{Lemat}
\newtheorem{fact}{Fakt}

\begin{document}

% streszczenie
\begin{abstract}
    
\end{abstract}

% słowa kluczowe
\keywords{automata}

% tytuł i spis treści
\maketitle

% wstęp
\introduction

Wstęp...

\chapter{Historia i definicja automatów komórkowych}
Automaty komórkowe zostały zaproponowane przez von Neumanna oraz Ulama i po raz pierwszy opisane w pracy Theory of Self-Reproducing Automata z 1966 roku.
Największy wkład w rozwój teorii automatów komórkowych miał Stephen Wolfram wydając szereg prac na ten temat w latach 80 dwudziestego wieku.

\section{Definicje i przykłady}
W dalszej części pracy będziemy rozważali automaty dwustanowe i przyjmujemy, że zbiór stanów to $\{0,1\}$.
\begin{definition}
 Niech $r \in \mathbb{N}$. Regułą lokalną nazywamy funkcję $f : \{0, 1 \}^{2r+1} \rightarrow \{0,1\}$.
\end{definition}
Automat komórkowy będziemy utożsamiać z regułą lokalną. Liczbę $r$ będzie nazywać promieniem sąsiedztwa reguły $f$.

\begin{definition}
 Niech $f$ jest regułą o promieniu $r$ oraz $N \in \mathbb{N}$. $N$-komórkową regułą globalną nazywamy funkcję 
 $F_{N} : \{0, 1 \}^{N} \rightarrow \{0,1\}^{N}$, spełniającą:
 \[
    F_{N}(...,x_{i},...) = (...,f(x_{i-r},...,x_{i},...,x_{i+r}),...),
 \]
oraz $x_{i+N} = x_{i}$, dla $i =0,1,...,N-1$.
\end{definition}

\begin{fact}
 Liczba wszystkich dwustanowych, jednowymiarowych automatów komórkowych o promieniu $r \in \mathbb{N}$ jest równa $2^{2^{2r+1}}$. 
\end{fact}

\begin{proof}
 Automat komórkowy utożsamiamy z regułą lokalną, zatem ich liczba jest równa liczbie wszystkich funkcji $f :\{0, 1 \}^{2r+1} \rightarrow \{0,1\}$.
 Liczba ta jest równa
 \[
  |\{0,1\}|^{|\{0,1\}^{2r+1}|} = 2^{2^{2r+1}}.
 \]

\end{proof}

W dalszym ciągu będziemy rozważali automaty komórkowe o promieniu $1$. Nazywamy je elementarnymi automatami komórkowymi i ich zbiór oznaczmy przez
ECA. Z faktu 1 wynika:

\begin{corollary}
 Wszystkich elementarnych automatów komórkowych jest 256, tj. $|ECA| = 256$.
\end{corollary}

Każdy automat komórkowy możemy jednoznacznie określić przez tabelę określającą wartości reguły lokalnej. Tabela ta nosi nazwę Lookup table.
\\Dla elementarnych automatów komórkowych ma ona postać:

\begin{center}
    \begin{tabular}{| l | l | l | l | l | l | l | l |}
    \hline
    111 & 110 & 101 & 100 & 011 & 010 & 001 & 000 \\ \hline
    $l_{7}$ & $l_{6}$ & $l_{5}$ & $l_{4}$ & $l_{3}$ & $l_{2}$ & $l_{1}$ & $l_{0}$ \\ \hline
    \end{tabular}
\end{center}
gdzie $l_{i} \in \{0,1\}$. Tabelę tę będziemy zapisywać skrótowo: $(l_{7}l_{6}l_{5}l_{4}l_{3}l_{2}l_{1}l_{0})$.
\\Każda reguła z klasy $ECA$ jest zatem jednoznacznie określona przez liczbę binarną $(l_{7}l_{6}l_{5}l_{4}l_{3}l_{2}l_{1}l_{0})$. W ten sposób
otrzymujemy numerację elementarnych automatów komórkowych: reguła $ECA_{i}$, gdzie $i \in \{0, 1, ..., 255\}$ to reguła, której wartości
Lookup table stanowią kolejne cyfry liczby $i$ w postaci binarnej.

\begin{example}
 Reguła $ECA_{0}$ jest reprezentowana przez Lookup table $(00000000)$, zaś reguła $ECA_{110}$ przez $(01101110)$.
\end{example}

\section{Klasyfikacje jednowymiarowych automatów komórkowych}
Automaty z klasy ECA można sklasyfikować ze względu na to jak przebiega ich ewolucja w czasie. Przy losowym warunku początkowym Wolfram zaproponował
klasyfikację na cztery grupy:
\enumerate
\item Klasa I - ewolucja automatu prowadzi do jednorodnego stanu, na przykład składającego się z samych 0
\item Klasa II - ewolucja prowadzi do stabilnego zbioru stanów lub do prostej okresowej struktru
\item Klasa III - ewolucja prowadzi do chaotycznego wzoru
\item Klasa IV - ewolucja prowadzi do złożonych struktur

Tak zaproponowano klasyfikacja nie pozwala jednak w niektórych przypadkach jednoznacznie sklasyfikować reguł z grup III i IV. Poniższa tabela prezentuje
klasyfikację dla reguł, które da się jednoznacznie sklasyfikować
\begin{center}
    \begin{tabular}{| l | l |}
    \hline
    Klasa I & 0, 8, 32, 40, 64, 96, 128, 136, 160, 168, 192, 224, 234, 235, 238, \\
	    &  239, 248, 249, 250, 251, 252, 253, 254, 255 \\ \hline
    Klasa II & 1, 2, 3, 4, 5, 6, 7, 9, 10, 11, 12, 13, 14, 15, 16, 17, 19, 20, 21, 23, 24 \\
	    &25, 26, 27, 28, 29, 31, 33, 34, 35, 36, 37, 38, 39, 42, 43, 44 \\
	    &46, 47, 48, 49, 50, 51, 52, 53, 55, 56, 57, 58, 59, 61, 62, 63, 65, 66 \\
          &67, 68, 69, 70, 71, 72, 73, 74, 76, 77, 78, 79, 80, 81, 82, 83, 84, 85 \\
          &87, 88, 91, 92, 93, 94, 95, 97, 98, 99, 100, 103, 104, 107, 108, 109, 111 \\
          &112, 113, 114, 115, 116, 117, 118, 119, 121, 123, 125, 127, 130, 131, 132 \\
          &133, 134, 138, 139, 140, 141, 142, 143, 144, 145, 148, 152, 154, 155, 156\\
          &157, 158, 159, 162, 163, 164, 166, 167, 170, 171, 172, 173, 174, 175, 176 \\
          &177, 178, 179, 180, 181, 184, 185, 186, 187, 188, 189, 190, 191, 194, 196 \\
          &197, 198, 199, 200, 201, 202, 203, 204, 205, 206, 207, 208, 209, 210, 211 \\
          &212, 213, 214, 215, 216, 217, 218, 219, 220, 221, 222, 223, 226, 227, 228 \\
          &229, 230, 231, 232, 233, 236, 237, 240, 241, 242, 243, 244, 245, 246, 247 \\ \hline
    Klasa III & 18, 22, 30, 45, 54, 60, 75, 86, 89, 90, 101, 102, 105, 122, 126, 129 \\
          & 135, 146, 149, 150, 151, 153, 161, 165, 182, 183, 195 \\ \hline
    Klasa IV & 41, 106, 110, 120, 124, 137, 147, 169, 193, 225 \\ \hline
    \end{tabular}
\end{center}

\chapter{Struktura zbioru dwustanowych automatów komórkowych}


\begin{definition}
 Niech $f \in ECA$. Funkcję $R : ECA \rightarrow ECA$ określoną wzorem
 \[
  Rf(x,y,z) = f(z,y,x),
 \]
 dla $x,y,z \in {0,1}$ nazywamy transformacją reflection.
 \\Funkcję $C : ECA \rightarrow ECA$ określoną wzorem
 \[
  Cf(x,y,z) = 1 - f(1-x,1-y,1-z),
 \]
 dla $x,y,z \in {0,1}$ nazywamy transformacją conjugate.
\end{definition}

\begin{fact}
 Niech $f \in ECA$. Złożenie $R\circ C$ wyraża się wzorem:
 \[
  RC f (x,y,z) = 1 - f(1-z, 1-y, 1-x)
 \]
dla $x,y,z \in {0,1}$ oraz spełniającą
\[
 RCf = CRf.
\]
\end{fact}

\begin{proof}
 Dla dowolnych $x,y,z \in \{0,1\}$:
 \[
  R(Cf(x,y,z)) = C(z,y,x) = 1 - f(1-z,1-y,1-x)
 \]
\[
 C(Rf(x,y,z)) = 1 - Rf(1-x, 1-y, 1-z) = 1 - f(1-z, 1-y, 1-x).
\]

\end{proof}

\begin{fact}
Dla $f \in ECA$ trasformacje $R$ i $C$ spełniają $RRf = f$ oraz $CCf = f$.
\end{fact}

\begin{proof}
 Dla dowolnych $x,y,z \in \{0,1\}$:
 \[
  R(Rf(x,y,z)) = Rf(z,y,x) = f(x,y,z),
 \]
 \[
  C(Cf(x,y,z)) = 1 - Cf(1-x,1-y,1-z) = 1 - (1 - f(1 - 1+x, 1 - 1+y, 1 - 1+z)) = f(x,y,z).
 \]

\end{proof}


\begin{fact}
 Niech $f \in ECA$ i $(l_{7}l_{6}l_{5}l_{4}l_{3}l_{2}l_{1}l_{0})$ to Lookup table dla reguły $f$. Wtedy
 \begin{enumerate}
  \item Lookup table dla reguły $Rf$ to 
  \[
   (l_{7}l_{3}l_{5}l_{1}l_{6}l_{2}l_{4}l_{0}).
  \]
  \item Lookup table dla reguły $Cf$ to 
  \[
  (1-l_{0}, 1- l_{1}, 1- l_{2}, 1- l_{3}, 1 - l_{4}, 1- l_{5}, 1 - l_{6}, 1 - l_{7}).
  \]
  \item Lookup table dla reguły $RCf$ to 
  \[
   (1-l_{0}, 1- l_{4}, 1- l_{2}, 1- l_{6}, 1 - l_{1}, 1- l_{5}, 1 - l_{3}, 1 - l_{7}).
  \]
 \end{enumerate}
\end{fact}

\begin{proof}
 Ponieważ $f(1,1,1) = l_{7},$ zatem $Rf(1,1,1) = f(1,1,1) = l_{7}$. Podobnie $Rf(1,1,0) = Rf(0,1,1) = l_{3}$. W ten sam sposób obliczamy 
 kolejne wartości podstawiając wszystkie $x,y,z \in \{0,1\}$.
 \\ Analogicznie wyznaczamy Lookup table dla $Cf$ oraz $RCf$.
\end{proof}

\begin{example}
 \begin{enumerate}
  \item Lookup table dla reguły $ECA_{0}$ to $(00000000)$. Stosując na niej transformację $R$, $C$ oraz $RC$ otrzymujemy kolejno $(00000000)$,
 $(11111111)$ oraz $(11111111)$. Zatem $ECA_{0} = R(ECA_{0})$ oraz \\$C(ECA_{0}) = RC(ECA_{0}) = ECA_{255}$.
 \item Dla reguły $ECA_{51}$ danej przez $(00110011)$, otrzymujemy $ECA_{51} = R(ECA_{51}) = C(ECA_{51}) = RC(ECA_{51})$.
 \item Dla reguły $ECA_{137}$ jej Lookup table to $(10001001)$. Dla transformacji $R$, $C$ oraz $RC$ otrzymujemy kolejno $(11000001)$, 
 $(01101110)$ oraz $(01111100)$, zatem $R(ECA_{137}) = ECA_{193}$, $C(ECA_{137}) = ECA_{110}$, $R(ECA_{137}) = ECA_{124}$.
 \end{enumerate}
\end{example}

Zbiór wszystkich elementarnych reguł $ECA$ możemy podzelić na rozłączne warstwy w ten sposób, że dwie reguły $f$ i $g$ należą do jednej warstwy wtedy
i tylko wtedy gdy $g$ możemy uzyskać stosując którąś z transformacji $R$, $C$ lub $RC$ na $f$. W przeciwnym wypadku będziemy mówili, że reguły
$f$ i $g$ są niezależne.

\begin{theorem}
 Liczba wszystkich niezależnych elementarnych automatów komórkowych jest równa $88$.
\end{theorem}

\begin{proof}
 Niech $f \in ECA$. Możliwe są następujące przypadki:
 \begin{enumerate}
  \item $Cf = Rf = f$. Wtedy $C(Rf) = Cf = f$.
  \item $Cf = Rf \neq f$. Wtedy $C(Rf) = CCf = f$.
  \item $Cf = f \neq Rf$. Wtedy $C(Rf) = R(Cf) = Rf$.
  \item $Cf \neq Rf = f$. Wtedy $C(Rf) = Cf$.
  \item $Cf \neq Rf \neq f$ oraz $Cf \neq f$. Gdyby $CRf = f$ to stosując do obu stron tej równości transformację $C$
  otrzymalibyśmy $CCRf = Cf$, skąd $Rf = Cf$, czyli sprzeczność z założeniem. Zatem $CRf \neq f$.
 \end{enumerate}
Zauważmy, że warstwa, której elementy spełniają warunek $1$ musi zawierać dokładniej jedną regułę, warstwa, która spełnia warunek $2$, $3$ albo
$4$ musi zawierać dokładnie dwie reguły, zaś warstwa której elementy spełniają warunek $5$ zawiera dokładnie cztery reguły.
\\Oznaczmy przez $n_{i}$ liczbę reguły spełniających warunek $i$, gdzie $i = 1,2,3,4,5$. Mamy oczywiście
\[
 n_{1} + n_{2} + n_{3} + n_{4} + n_{5} = 256.
\]
Należy pokazać, że
\[
 n_{1} + \frac{n_{2}}{2} + \frac{n_{3}}{2} + \frac{n_{4}}{2} + \frac{n_{5}}{4} = 88.
\]
Niech $(l_{7}l_{6}l_{5}l_{4}l_{3}l_{2}l_{1}l_{0})$ to Lookup table dla reguły $f \in ECA$. Porównując wartości Lookup table dla reguł $f$, $Rf$
oraz $Cf$ dla każdego z powyższych przypadków otrzymujemy:
\begin{enumerate}
 \item $l_{7} = 1 - l_{0}$, $l_{2} = 1 - l_{5}$, $l_{1} = l_{4} = 1 - l_{6}$ oraz $l_{3} = l_{6}$. Zatem Lookup table dla reguł spełniającej 
 ten warunek zależna jest tylko od $3$ liczb. Stąd $n_{1} = 2^{3} = 8$.
 \item Z warunku $Cf = Rf$ otrzymujemy $l_{7} = 1 - l_{0}$, $l_{3} = 1 - l_{1}$, $l_{5} = 1 - l_{2}$ oraz $l_{4} = 1- l_{6}$. Wszystkich reguł
 spełniających ten warunek jest $2^{4}$. Wykluczając reguły spełniające $Rf = Cf = f$ (czyli warunek $1$) otrzymujemy, że $n_{2} = 16 - 8 = 8$.
 \item Z warunku $Cf = f$ otrzymujemy $l_{7} = 1 - l_{0}$, $l_{6} = 1 - l_{1}$, $l_{5} = 1 - l_{2}$ oraz $l_{4} = 1- l_{3}$. Wszystkich reguł
 spełniających ten warunek jest $2^{4}$. Wykluczając reguły warunek $1$ otrzymujemy, że $n_{3} = 16 - 8 = 8$.
 \item Z warunku $Rf = f$ otrzymujemy $l_{4} = l_{1}$ oraz $l_{3} = l_{6}$. Wszystkich reguł spełniających ten warunek jest $2^{6}$. 
 Wykluczając czyli warunek $1$ otrzymujemy, że $n_{4} = 64 - 8 = 56$.
 \item Liczbę reguł spełniających ten warunek obliczmy z równości na liczbę wszystkich reguł $ECA$:
 \[
  n_{5} = 256 - n_{1} - n_{2} - n_{3} - n_{4} = 256 - 8 - 8 - 8 - 56 = 176.
 \]
\end{enumerate}
 Zatem
 \[
    n_{1} + \frac{n_{2}}{2} + \frac{n_{3}}{2} + \frac{n_{4}}{2} + \frac{n_{5}}{4} = 8 + 4 + 4 + 28 + 44 = 88,
 \]
co należało pokazać.

\end{proof}

\begin{corollary}
 Liczba reguł należących do jednej warstwy wynosi 1, 2 albo 4.
\end{corollary}

\chapter{Ogólne własności elementarnych automatów komórkowych}
W niniejszym rozdziale zbadamy niektóre własności jednowymiarowych automatów komórkowych. Oblicznia będziemy prowadzili dla wektora 
początkowego o długości 8 posiadającego każde z możliwych sąsiedztw:
\[
 I = (11101000).
\]
Można wykazać, że dla automatu o promieniu $r\in\mathbb{N}$ istnieje wektor posiadający każde z $2^{2r+1}$ sąsiedztw.
\section{Stosunki sąsiedztw}
Niech $T > 0$ oraz $TSD \in \{0,1\}^{(T+1)\times8}$ to macierz będąca space-time diagram 

\chapter{Rozkłady stosunków sąsiedztw}
W niniejszym rozdziale zbadamy jakie są rozkłady sąsiedztw dla wektorów początkowych o długości większej niż 8.

\section{Reguła 184}

\begin{theorem}
 Niech $N$ to długość konfiguracji początkowej i $T = \lceil N \rceil $. Wtedy 
 \begin{enumerate}
  \item Jeśli $D(S(0)) > 0.5$ to istnieje $i \in \{0, 1, ..., N - 1 \}$ takie, że
 \end{enumerate}

\end{theorem}

% zakończenie
\summary

% załączniki (opcjonalnie):
\appendix
\chapter{Tytuł załącznika jeden}

Treść załącznika jeden.

\chapter{Tytuł załącznika dwa}

Treść załącznika dwa.

% literatura (obowiązkowo):
\bibliographystyle{unsrt}
\bibliography{xml}

% spis tabel (jeżeli jest potrzebny):
\listoftables

% spis rysunków (jeżeli jest potrzebny):
\listoffigures

\oswiadczenie

\end{document}
